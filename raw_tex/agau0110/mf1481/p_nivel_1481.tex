\documentclass[11pt]{exam}
\usepackage[spanish]{babel}
\usepackage[utf8]{inputenx}
\usepackage{fontenc}
\usepackage{textcomp}
\usepackage{lmodern,pifont}
\usepackage{graphicx}
\graphicspath{ {./images_1481/} }
\usepackage{setspace}
\usepackage[dvipsnames]{color}
\usepackage{colortbl}
\usepackage{caption}
\usepackage{amsmath}
\usepackage[normalem]{ulem}

\newcommand\titexam[1]{\centering%
\fbox{\parbox{\textwidth}{\huge \sffamily \textbf{#1}}}\normalsize \vspace{1em}}

\newcommand\materia[1]{%
\parbox{\textwidth}{ \Large \sffamily \textbf{\uline{#1}}}\vspace{1em}}

\newcommand\nombrefecha{%
Nombre y apellidos:\hrulefill
Fecha:\rule{3.5cm}{0.4pt}\vspace{0.5em}}

\renewcommand{\solutiontitle}{\noindent\textbf{Solución:}\par\noindent}
\pagestyle{empty}
\begin{document}
{\fontfamily{lmss}\selectfont

  %%%%%%%%%%%%%%%%%%%%%%%%%%%%%%%%%%%%%
  %% 
\titexam{Producción de semillas}

\nombrefecha

\materia{Características de los frutos y semillas}
\begin{questions}
%1
\question La frase: ``El fruto es una parte de los árboles que sirve para atraer
a diferentes animales y asegura que se fecunden mas flores''
\begin{checkboxes}
  \choice A. Verdadera
  \CorrectChoice B. Falsa
  \choice C. Falsa ya que no sirve para atraer a diferentes animales, solo
  asegura la fecundación
  \choice D. Falsa ya que no sirve para asegurar la fecundación, solo atrae a
  diferentes animales
\end{checkboxes}
%%2
\question ¿Todos los frutos han de formarse \uline{únicamente} despues de la
fecundación del óvulo?
\begin{checkboxes}
  \CorrectChoice A. No es correcta. Hay plantas en las que se da el fenómeno de
  la partenocarpia
  \choice B. No es correcta ya que hay plantas que tienen fecundación autogama
  \choice C. Es correcta. No hay frutos que puedan formarse sin la fecundación
  previa del óvulo
  \choice D. Es correcta solo si las plantas tienen dispersión del polen por
  medio de insectos
\end{checkboxes}
%%3
\question ¿Las partes de un fruto son?
\begin{checkboxes}
  \choice A. Embrion, endospermo, epispermo
  \CorrectChoice B. Epispermo, mesocarpio, endocarpio
  \choice C. Radicula, plumula e hipocólito
  \choice D. Ninguna es correcta
\end{checkboxes}
%%4
\question ¿El cotiledon es?
\begin{checkboxes}
  \choice A. Parte de una planta que está cerca de la raíz
  \choice B. Las hojas que se encuentran en el ápice
  \choice C. La capa exterior de una semilla
  \CorrectChoice D. Ninguna respuesta es correcta
\end{checkboxes}
%% 5
\question Las semillas con el tiempo disminuyen su capacidad de germinar;
decimos entonces que disminuye su?
\begin{checkboxes}
  \CorrectChoice A. Viabilidad
  \choice B. Vigor
  \choice C. Resistencia
  \choice D. Longevidad
\end{checkboxes}
%% 6
\question Se define el vigor de una semilla como ``un conjunto de propiedades
que determinan el nivel de actividad y capacidad de las semillas durante la
germinación y durante la emergencia de las plántulas''. ¿Esta propiedad es
resultado de? 
\begin{checkboxes}
  \choice A. Constitución genética
  \choice B. Condiciones ambientales
  \choice C. Grado de deterioro y envejecimiento
  \CorrectChoice D. Todas las respuestas son correctas
\end{checkboxes}
%% 7 
\question ¿La longevidad de una semilla se define como?
\begin{checkboxes}
  \choice A. Capacidad de germinar y originar plantulas normales en condiciones
  favorables
  \choice B. Capacidad de permanecer funcional despues de ser secadas
  \CorrectChoice C. Tiempo por el que pueden mantenerse viables en unas
  determinadas condiciones de temperatura y contenido de humedad
  \choice D. La capacidad de producir plantulas con elevadas tasas de crecimiento
\end{checkboxes}
%% 8
\question Según el contenido de humedad con el que podemos almacenar las
semillas, ¿distinguimos?
\begin{checkboxes}
  \choice A. Semillas ortodoxas
  \choice B. Semillas longevas y desecantes
  \choice C. Semillas recalcitrantes
  \CorrectChoice D. Las respuestas A y C son correctas
\end{checkboxes}
%% 9
\question ¿Como se llaman las semillas que pueden ser almacenadas después des
ser desecadas en condiciones optimas de viabilidad?
\begin{checkboxes}
  \choice A. Semillas secas
  \choice B. Semillas recalcitrantes
  \CorrectChoice C. Semillas ortodoxas
  \choice D. Ninguna respuesta es correcta 
\end{checkboxes}
%% 10
\question ¿Las gimmnospermas son?
\begin{checkboxes}
  \choice A. Todas las plantas que tienen flor
  \CorrectChoice B. Las plantas que no tienen flores verdaderas, y sus semillas
  se desarrollan en conos. Pinos, cedros, cipreses, sabinas, etc
  \choice C. Las plantas que se fecundan solas
  \choice D. Los árboles que tienen hojas estrechas y alargadas que parecen
  agujas 
\end{checkboxes}
\end{questions}
\newpage
%%%%%%%%%%%%%%%%%%%%%%%%%%%%%%%%%%%%%%%%%%%%%%%%%%%%%%%%%%%%%%%%%%%%%%%%%%%%%%%%%%%%%%%%%%
%%% Recolección de frutos y semillas
\titexam{Producción de semillas}

\nombrefecha

\materia{Recolección de frutos y semillas}
\begin{questions}
% 1 
\question ¿Según la forma de liberar las semillas los frutos los clasificamos
  en?
  \begin{checkboxes}
    \CorrectChoice A. Indehiscentes y Dehiscentes
    \choice B. Monospermos o polispermos
    \choice C. Carnosos o secos
    \choice D. Ninguna respuesta es correcta
  \end{checkboxes}
  % 2
\question ¿Qué plantas son las que producen crecimiento vegetativo durante la
  estación de crecimiento, tienen un lento bajón con el tiempo frío, van a
  semilla en la segunda estación de crecimiento y entonces mueren?
  \begin{checkboxes}
    \CorrectChoice A. Plantas bianuales
    \choice B. Plantas leñosas
    \choice C. PLantas perennes
    \choice D. Las respuestas A y C son correctas
  \end{checkboxes}
  % 3
\question ¿Cuando cosechamos fruytos para recoger semillas de especies como el
  tomate (\emph{Solanum lycopersicum})?
  \begin{checkboxes}
    \choice A. Los dejamos secar y después los recogemos
    \choice B. Los recogemos a medida maduran
    \CorrectChoice C. Dejamos que el fruto madure lo más posible en la mata 
    \choice D. A finales de verano
  \end{checkboxes}
  % 4
\question Las plantas que tenen tendencia a la dehiscencia como lechugas,
  zanahorias o cebollas, en general ¿cuando hay que recoger la semilla?
  \begin{checkboxes}
    \choice A. Los dejamos secar y después los recogemos
    \CorrectChoice B. Los recogemos a medida maduran
    \choice C. Dejamos que el fruto madure lo más posible en la mata
    \choice D. A finales de verano
  \end{checkboxes}
  % 5
\question ¿En que tipo de recipiente voy a recolectar semillas con aristas o
  frutos con ganchos?
  \begin{checkboxes}
    \choice A. En bolsas de plástico
    \choice B. En bolsas de tela
    \CorrectChoice C. En bolsas de papel
    \choice D. En baldes
  \end{checkboxes}
  % 6
\question ¿Qué es una panícula?
  \begin{checkboxes}
    \choice A. Un método de cosecha
    \choice B. Una rama de una mata con flores
    \CorrectChoice C. Un racimo ramificado de flores en el que las ramas son a
    su vez racimos
    \choice D. Ninguna respuesta es correcta 
  \end{checkboxes}
  % 7
\question ¿Los sistemas mecánicos de recolección los clasificamos en?
  \begin{checkboxes}
    \CorrectChoice A. Vibración y sacudida
    \choice B. Manuales o acoplados al tractor
    \choice C. Autopropulsados o acoplados
    \choice D. Manuales o cabalgantes
  \end{checkboxes}
  % 8
\question Si empleamos un método para la recogida de frutos de árboles como
  agitar las ramas, ¿qué tipo de utensilios debería emplear para recogerlos?
  \begin{checkboxes}
    \choice A. Baldes de gran tamaño
    \choice B. Bolsas de papel
    \choice C. Bolsas de plástico
    \CorrectChoice D. Ninguna respuesta es correcta
  \end{checkboxes}
  % 9
\question ¿En que casos empleariamos una técnica de recogida como las cosecha de
  frutos enteros de forma manual?
  \begin{checkboxes}
    \choice A. Cuando no podemos realizar una selección y descarte de una manera
    más eficiente
    \choice B. La accesibilidad de los frutos permite emplear las dos manos para dejar las semillas en
    un cubo, balde u otro recipiente
    \choice C. Los frutos contienen un alto número de semillas, sean carnosos o
    secos indehiscentes
    \CorrectChoice D. Todas las respuestas son correctas 
  \end{checkboxes}
  % 10
\question ¿Cuando las semillas forman parte de la parte comestible de la planta
  qué es lo que conviene hacer con ellas para su reclección?
  \begin{checkboxes}
    \choice A. Recogerlos muy maduros y cuando se han empezado a ablandar
    \CorrectChoice B. Se pueden dejar en la planta hasta que estén completamente
    secas, previendo que el tiempo y los animales no las estropeen
    \choice C. Los recogemos a medida maduran
    \choice D. Los podemos recoger en invierno
  \end{checkboxes}
\end{questions}
\newpage
%%%%%%%%%%%%%%%%%%%%%%%%%%%%%%%%%%%%%%%%%%%%%%%%%%%%%%%%%%%%%%%%%%%%%%%%%%%%%%%%%%%%%%%%%%
%%% Recolección de frutos y semillas
\titexam{Producción de semillas}

\nombrefecha

\materia{Preparación y acondicionamiento de lotes de frutos y semillas}
\begin{questions}
  % 1
  
\question ¿Cuando recogemos frutos en campo, se corre el riesgo de que la semilla
  se deteriore y?
  \begin{checkboxes}
    \choice A. Que se pierda la identidad del lote
    \choice B. Que la viabilidad del lote se reduzca
    \choice C. Qué con buena ventilación se deterioren los frutos
    \CorrectChoice D. Las respuestas A y B son correctas
  \end{checkboxes}
  % 2
\question ¿El procesamiento de las semillas lo realizaremos preferentemente en?
  \begin{checkboxes}
    \choice A. Siempre en el lugar de cosecha del fruto
    \CorrectChoice B. Si las condiciones topográficas o climáticas no son
    favorables lo realizaremos en las instalaciones
    \choice 
  \end{checkboxes}
\end{questions}
}
\end{document}
%%% Local Variables:
%%% mode: latex
%%% TeX-master: t
%%% End:
