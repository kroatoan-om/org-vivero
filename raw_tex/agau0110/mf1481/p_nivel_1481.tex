\documentclass[11pt,answers]{exam}
\usepackage[spanish]{babel}
\usepackage[utf8]{inputenx}
\usepackage{fontenc}
\usepackage{textcomp}
\usepackage{lmodern,pifont}
\usepackage{graphicx}
\graphicspath{ {./images/} }
\usepackage{setspace}
\usepackage[dvipsnames]{color}
\usepackage{colortbl}
\usepackage{caption}
\usepackage{amsmath}
\usepackage[normalem]{ulem}

\newcommand\titexam[1]{\centering%
\fbox{\parbox{\textwidth}{\huge \sffamily \textbf{#1}}}\normalsize \vspace{1em}}

\newcommand\materia[1]{%
\parbox{\textwidth}{ \Large \sffamily \textbf{\uline{#1}}}\vspace{1em}}

\newcommand\nombrefecha{%
Nombre y apellidos:\hrulefill
Fecha:\rule{3.5cm}{0.4pt}\vspace{0.5em}}

\renewcommand{\solutiontitle}{\noindent\textbf{Solución:}\par\noindent}
\pagestyle{empty}
\begin{document}
{\fontfamily{lmss}\selectfont

  %%%%%%%%%%%%%%%%%%%%%%%%%%%%%%%%%%%%%
  %% 
\titexam{Producción de semillas}

\nombrefecha

\materia{Características de lls frutos y semillas}
\begin{questions}
%1
\question La frase: ``El fruto es una parte de los árboles que sirve para atraer
a diferentes animales y asegura que se fecunden mas flores''
\begin{checkboxes}
  \choice A. Verdadera
  \CorrectChoice B. Falsa
  \choice C. Falsa ya que no sirve para atraer a diferentes animales, solo
  asegura la fecundación
  \choice D. Falsa ya que no sirve para asegurar la fecundación, solo atrae a
  diferentes animales
\end{checkboxes}
%%2
\question ¿Todos los frutos han de formarse \uline{unicamente} despues de la
fecundación del óvulo?
\begin{checkboxes}
  \CorrectChoice A. No es correcta. Hay plantas en las que se da el fenómeno de
  la partenocarpia
  \choice B. No es correcta ya que hay plantas que tienen fecundación autogama
  \choice C. Es correcta. No hay frutos que puedan formarse sin la fecundación
  previa del óvulo
  \choice D. Es correcta solo si las plantas tienen dispersión del polen por
  medio de insectos
\end{checkboxes}
%%3
\question ¿Las partes de un fruto son?
\begin{checkboxes}
  \choice A. Embrion, endospermo, epispermo
  \CorrectChoice B. Epispermo, mesocarpio, endocarpio
  \choice C. Radicula, plumula e hipocólito
  \choice D. Ninguna es correcta
\end{checkboxes}
%%4
\question ¿El cotiledon es?
\begin{checkboxes}
  \choice A. Parte de una planta que está cerca de la raíz
  \choice B. Las hojas que se encuentran en el ápice
  \choice C. La capa exterior de una semilla
  \CorrectChoice D. Ninguna respuesta es correcta
\end{checkboxes}
%% 5
\question Las semillas con el tiempo disminuyen su capacidad de germinar;
decimos entonces que disminuye su?
\begin{checkboxes}
  \CorrectChoice A. Viabilidad
  \choice B. Vigor
  \choice C. Resistencia
  \choice D. Longevidad
\end{checkboxes}
%% 6
\question Se define el vigor de una semilla como ``un conjunto de propiedades
que determinan el nivel de actividad y capacidad de las semillas durante la
germinación y durante la emergencia de las plántulas''. ¿Esta propiedad es
resultado de? 
\begin{checkboxes}
  \choice A. Constitución genética
  \choice B. Condiciones ambientales
  \choice C. Grado de deterioro y envejecimiento
  \CorrectChoice D. Todas las respuestas son correctas
\end{checkboxes}
%% 7 
\question ¿La longevidad de una semilla se define como?
\begin{checkboxes}
  \choice A. Capacidad de germinar y originar plantulas normales en condiciones
  favorables
  \choice B. Capacidad de permanecer funcional despues de ser secadas
  \CorrectChoice C. Tiempo por el que pueden mantenerse viables en unas
  determinadas condiciones de temperatura y contenido de humedad
  \choice D. La capacidad de producir plantulas con elevadas tasas de crecimiento
\end{checkboxes}
%% 8
\question Según el contenido de humedad con el que podemos almacenar las
semillas, ¿distinguimos?
\begin{checkboxes}
  \choice A. Semillas ortodoxas
  \choice B. Semillas longevas y desecantes
  \choice C. Semillas recalcitrantes
  \CorrectChoice D. Las respuestas A y C son correctas
\end{checkboxes}
%% 9
\question ¿Como se llaman las semillas que pueden ser almacenadas después des
ser desecadas en condiciones optimas de viabilidad?
\begin{checkboxes}
  \choice A. Semillas secas
  \choice B. Semillas recalcitrantes
  \CorrectChoice C. Semillas ortodoxas
  \choice D. Ninguna respuesta es correcta 
\end{checkboxes}
%% 10
\question ¿Las gimmnospermas son?
\begin{checkboxes}
  \choice A. Todas las plantas que tienen flor
  \CorrectChoice B. Las plantas que no tienen flores verdaderas, y susu semillas
  se desarrollan en conos. Pinos, cedros, cipreses, sabinas, etc
  \choice C. Las plantas que se fecundan solas
  \choice D. Los árboles que tienen hojas estrechas y alargadas que parecen
  agujas 
\end{checkboxes}
\end{questions}}
\end{document}
%%% Local Variables:
%%% mode: latex
%%% TeX-master: t
%%% End:
