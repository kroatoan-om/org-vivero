\documentclass[11pt]{exam}
\usepackage[spanish]{babel}
\usepackage[utf8]{inputenx}
\usepackage{fontenc}
\usepackage{textcomp}
\usepackage{lmodern,pifont}
\usepackage{graphicx}
\usepackage{setspace}
\usepackage[dvipsnames]{color}
\usepackage{colortbl}
\usepackage{caption}
\usepackage{amsmath}

\renewcommand{\solutiontitle}{\noindent\textbf{Solución:}\par\noindent}
\pagestyle{empty}
\begin{document}
{\fontfamily{lmss}\selectfont
\textbf{INSTRUCCIONES:}
\begin{itemize}
    \item Se dispone de dos horas para responder las 20 preguntas.
    \item Cada pregunta tiene un valor de 0.5 punto.
    \item Para considerar la respuesta como correcta, la opción escogida ha de
      estar correctamente señalada. Las preguntas erróneamente marcadas se
      considerarán como incorrectas. 
    \item Cada respuesta incorrecta resta 0.17 puntos. Las respuestas en blanco no restan. 
\end{itemize}
\vspace{1cm}
  \begin{questions}
    % 1
\question Se define el vigor de una semilla como ``un conjunto de propiedades
que determinan el nivel de actividad y capacidad de las semillas durante la
germinación y durante la emergencia de las plántulas''. ¿Esta propiedad es
resultado de? 
\begin{checkboxes}
  \choice A. Constitución genética
  \choice B. Condiciones ambientales
  \choice C. Grado de deterioro y envejecimiento
  \CorrectChoice D. Todas las respuestas son correctas
\end{checkboxes}
% 2
\question ¿Todos los frutos han de formarse \uline{únicamente} después de la
fecundación del óvulo?
\begin{checkboxes}
  \CorrectChoice A. No es correcta. Hay plantas en las que se da el fenómeno de
  la partenocarpia
  \choice B. No es correcta ya que hay plantas que tienen fecundación autógena
  \choice C. Es correcta. No hay frutos que puedan formarse sin la fecundación
  previa del óvulo
  \choice D. Es correcta solo si las plantas tienen dispersión del polen por
  medio de insectos
\end{checkboxes}
% 3
\question Según el contenido de humedad con el que podemos almacenar las
semillas, ¿distinguimos?
\begin{checkboxes}
  \choice A. Semillas ortodoxas
  \choice B. Semillas longevas y desecantes
  \choice C. Semillas recalcitrantes
  \CorrectChoice D. Las respuestas A y C son correctas
\end{checkboxes}
% 4
\question Las plantas que tienen tendencia a la dehiscencia como lechugas,
  zanahorias o cebollas, en general ¿cuando hay que recoger la semilla?
  \begin{checkboxes}
    \choice A. Los dejamos secar y después los recogemos
    \CorrectChoice B. Los recogemos a medida maduran
    \choice C. Dejamos que el fruto madure lo más posible en la mata
    \choice D. A finales de verano
  \end{checkboxes}
  \newpage
% 5
\question ¿Qué es una panícula?
  \begin{checkboxes}
    \choice A. Un método de cosecha
    \choice B. Una rama de una mata con flores
    \CorrectChoice C. Un racimo ramificado de flores en el que las ramas son a
    su vez racimos
    \choice D. Ninguna respuesta es correcta 
  \end{checkboxes}
  % 6
\question ¿En que tipo de recipiente voy a recolectar semillas con aristas o
  frutos con ganchos?
  \begin{checkboxes}
    \choice A. En bolsas de plástico
    \choice B. En bolsas de tela
    \CorrectChoice C. En bolsas de papel
    \choice D. En baldes
  \end{checkboxes}
  % 7
\question ¿Cuando las semillas forman parte de la parte comestible de la planta
  qué es lo que conviene hacer con ellas para su recolección?
  \begin{checkboxes}
    \choice A. Recogerlos muy maduros y cuando se han empezado a ablandar
    \CorrectChoice B. Se pueden dejar en la planta hasta que estén completamente
    secas, previendo que el tiempo y los animales no las estropeen
    \choice C. Los recogemos a medida maduran
    \choice D. Los podemos recoger en invierno
  \end{checkboxes}
  % 8
\question La frase: ``El fruto es una parte de los árboles que sirve para atraer
a diferentes animales y asegura que se fecunden mas flores''
\begin{checkboxes}
  \choice A. Verdadera
  \CorrectChoice B. Falsa
  \choice C. Falsa ya que no sirve para atraer a diferentes animales, solo
  asegura la fecundación
  \choice D. Falsa ya que no sirve para asegurar la fecundación, solo atrae a
  diferentes animales
\end{checkboxes}
% 9
\question ¿Las partes de un fruto son?
\begin{checkboxes}
  \choice A. Embrión, endospermo, epispermo
  \CorrectChoice B. Epispermo, mesocarpio, endocarpio
  \choice C. Radícula, plúmula e hipocólito
  \choice D. Ninguna es correcta
\end{checkboxes}
% 10
\question ¿Según la forma de liberar las semillas los frutos los clasificamos
  en?
  \begin{checkboxes}
    \CorrectChoice A. Indehiscentes y Dehiscentes
    \choice B. Monospermos o polispermos
    \choice C. Carnosos o secos
    \choice D. Ninguna respuesta es correcta
  \end{checkboxes}
  \newpage
  % 11
\question ¿Los sistemas mecánicos de recolección los clasificamos en?
  \begin{checkboxes}
    \CorrectChoice A. Vibración y sacudida
    \choice B. Manuales o acoplados al tractor
    \choice C. Autopropulsados o acoplados
    \choice D. Manuales o cabalgantes
  \end{checkboxes}
  % 12
\question ¿En que casos emplearíamos una técnica de recogida como las cosecha de
  frutos enteros de forma manual?
  \begin{checkboxes}
    \choice A. Cuando no podemos realizar una selección y descarte de una manera
    más eficiente
    \choice B. La accesibilidad de los frutos permite emplear las dos manos para dejar las
    semillas en un cubo, balde u otro recipiente 
    \choice C. Los frutos contienen un alto número de semillas, sean carnosos o
    secos indehiscentes
    \CorrectChoice D. Todas las respuestas son correctas 
  \end{checkboxes}
  % 13
\question Las mesas de gravedad son un método de separación y limpieza mecanizada. ¿Qué
  característica de las semillas emplean como principio de funcionamiento?
  \begin{checkboxes}
    \choice A. La forma de las semillas
    \CorrectChoice B. El peso específico de las semillas
    \choice C. La textura superficial de las semillas
    \choice D. El grado de humedad de las semillas
  \end{checkboxes}
  % 14
\question ¿Una buena medida de precaución para el etiquetado de lotes de frutos y semillas
  en campo es?
  \begin{checkboxes}
    \choice A. Guardar las etiquetas en bolsas de plástico separadas para que no se
    deterioren
    \CorrectChoice B. Etiquetar por la parte exterior y la inferior del recipiente
    \choice C. Grapar etiquetas plastificadas
    \choice D. Las respuestas A y B son correctas
  \end{checkboxes}
  % 15
\question Los objetivos que se persiguen con el acondicionamiento de semillas son:
  eliminar el exceso de humedad y materiales indeseables, clasificar adecuadamente las
  semillas ¿y?
  \begin{checkboxes}
    \CorrectChoice A. Proteger las semillas contra plagas y enfermedades
    \choice B. Lograr una comercialización barata
    \choice C. Mantener las perdidas de semillas al mínimo
    \choice D. Conseguir que su contenido de humedad sea el adecuado
  \end{checkboxes}
  % 16
\question ¿Los métodos de extracción los clasificamos en?
  \begin{checkboxes}
    \choice A. Aventado y cribado
    \choice B. Siega y trillado
    \CorrectChoice C. Secos y húmedos
    \choice D. Húmedos y despulpado
  \end{checkboxes}
  \newpage
  % 17
\question ¿Como se llama a la operación de ``separar el grano de la paja'' mediante la
  rotura de las infrutescencias?
  \begin{checkboxes}
    \choice A. Extracción en seco
    \choice B. Decantación húmeda
    \CorrectChoice C. Trillado
    \choice D. Ninguna respuesta es correcta 
  \end{checkboxes}
  % 18
\question En general para conservar las semillas debemos mantener un ambiente frío, limpio
  y seco. ¿Qué otras recomendaciones son importantes tener en cuenta para el
  almacenamiento de semillas?
  \begin{checkboxes}
    \choice A. Conservarlas en estado latente para preservar su viabilidad
    \choice B. Mantener una mínima diferencia entre la humedad del ambiente y de las
    semillas
    \choice C. Las semillas que poseen bajos niveles de humedad y que son conservadas a
    temperaturas bajas germinan antes
    \CorrectChoice D. Las respuestas A y B son correctas
  \end{checkboxes}
  % 19
\question ¿Cuando cosechamos frutos para recoger semillas de especies como el
  tomate (\emph{Solanum lycopersicum})?
  \begin{checkboxes}
    \choice A. Los dejamos secar y después los recogemos
    \choice B. Los recogemos a medida maduran
    \CorrectChoice C. Dejamos que el fruto madure lo más posible en la mata 
    \choice D. A finales de verano
  \end{checkboxes}
  % 20
\question ¿El aventado es?
  \begin{checkboxes}
    \choice A. Un método de extracción en seco
    \choice B. Un método de selección y clasificación
    \choice C. Emplear el viento para separar impurezas para obtener las semillas
    \CorrectChoice D. Las respuestas A y C son correctas
  \end{checkboxes}
\end{questions}}
\end{document}

%%% Local Variables:
%%% mode: latex
%%% TeX-master: t
%%% End:
