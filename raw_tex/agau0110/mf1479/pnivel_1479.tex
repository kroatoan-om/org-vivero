\documentclass[11pt]{exam}
\usepackage[spanish]{babel}
\usepackage[utf8]{inputenx}
\usepackage{fontenc}
\usepackage{textcomp}
\usepackage{lmodern,pifont}
\usepackage{graphicx}
\graphicspath{ {./images/} }
\usepackage{setspace}
\usepackage[dvipsnames]{color}
\usepackage{colortbl}
\usepackage{caption}
\usepackage{amsmath}
\usepackage[normalem]{ulem}

\newcommand\titexam[1]{\centering%
\fbox{\parbox{\textwidth}{\huge \sffamily \textbf{#1}}}\normalsize \vspace{1em}}

\newcommand\materia[1]{%
\parbox{\textwidth}{ \Large \sffamily \textbf{\uline{#1}}}\vspace{1em}}


\newcommand\nombrefecha{%
Nombre y apellidos:\hrulefill
Fecha:\rule{3.5cm}{0.4pt}\vspace{0.5em}}

\renewcommand{\solutiontitle}{\noindent\textbf{Solución:}\par\noindent}
\pagestyle{empty}
\begin{document}
{\fontfamily{lmss}\selectfont

  %%%%%%%%%%%%%%%%%%%%%%%%%%%%%%%%%%%%%
  %% 
\titexam{MF1479\_2 Propagación de plantas en vivero}

\nombrefecha

\materia{Aspectos básicos de botánica y ecofisiología vegetal}
\begin{questions}
\question El nombre científico de una especie se forma de dos partes. ¿Puedes
indicar a que categoría taxonómica corresponde la primera parte?
\begin{checkboxes}
  \choice A. Al Reino
  \choice B. A la Família
  \CorrectChoice C. Al Género
  \choice D. Ninguna es correcta
\end{checkboxes}

\question ¿Sabes que terminación han de tener los taxones pertenecientes a la
categoría de la familia?
\begin{checkboxes}
  \CorrectChoice A. A la Família
  \choice B. Al Genero
  \choice C. Al Orden
  \choice D. A la especie
\end{checkboxes}

\question Señala cuales son coníferas
\begin{checkboxes}
  \choice A. Plataneros, encinas y robles
  \choice B. Pinos, cedros y abetos
  \choice C. Enebros, sabinas y cipreses
  \CorrectChoice D. Las respuestas B y C son correctas
\end{checkboxes}
\end{questions}}
\end{document}
%%% Local Variables:
%%% mode: latex
%%% TeX-master: t
%%% End:
