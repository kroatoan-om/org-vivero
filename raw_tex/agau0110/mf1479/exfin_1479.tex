\documentclass[11pt]{exam}
\usepackage[spanish]{babel}
\usepackage[utf8]{inputenx}
\usepackage{fontenc}
\usepackage{textcomp}
\usepackage{lmodern,pifont}
\usepackage{graphicx}
\usepackage{setspace}
\usepackage[dvipsnames]{color}
\usepackage{colortbl}
\usepackage{caption}
\usepackage{amsmath}

\renewcommand{\solutiontitle}{\noindent\textbf{Solución:}\par\noindent}
\pagestyle{empty}
\begin{document}
{\fontfamily{lmss}\selectfont
\textbf{INSTRUCCIONES:}
\begin{itemize}
    \item Se dispone de dos horas para responder las 20 preguntas. 10 por cada unidad formativa
    \item Cada pregunta tiene un valor de 1 punto.
    \item Para aprobar el modulo es necesario aprobar cada unidad formativa,
      esto es, se ha de obtener una puntuación mínima de 5 puntos en cada unidad
      formativa 
    \item Para considerar la respuesta como correcta, la opción escogida ha de
      estar correctamente señalada. Las preguntas erroneamente marcadas se
      considerarán como incorrectas. 
    \item Cada respuesta incorrecta resta 0.33 puntos. Las respuestas en blanco no restan. 
\end{itemize}
\vspace{1cm}
\textbf{UF 0009. Mantenimiento, preparación y manejo de tractores}
\begin{questions}
\question 

\end{questions}}

\end{document}
%%% Local Variables:
%%% mode: latex
%%% TeX-master: t
%%% End:
