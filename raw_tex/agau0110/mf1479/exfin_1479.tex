\documentclass[11pt]{exam}
\usepackage[spanish]{babel}
\usepackage[utf8]{inputenx}
\usepackage{fontenc}
\usepackage{textcomp}
\usepackage{lmodern,pifont}
\usepackage{graphicx}
\usepackage{setspace}
\usepackage[dvipsnames]{color}
\usepackage{colortbl}
\usepackage{caption}
\usepackage{amsmath}

\renewcommand{\solutiontitle}{\noindent\textbf{Solución:}\par\noindent}
\pagestyle{empty}
\begin{document}
{\fontfamily{lmss}\selectfont
\textbf{INSTRUCCIONES:}
\begin{itemize}
    \item Se dispone de dos horas para responder las 20 preguntas. 10 por cada unidad formativa
    \item Cada pregunta tiene un valor de 1 punto.
    \item Para aprobar el modulo es necesario aprobar cada unidad formativa,
      esto es, se ha de obtener una puntuación mínima de 5 puntos en cada unidad
      formativa 
    \item Para considerar la respuesta como correcta, la opción escogida ha de
      estar correctamente señalada. Las preguntas erroneamente marcadas se
      considerarán como incorrectas. 
    \item Cada respuesta incorrecta resta 0.33 puntos. Las respuestas en blanco no restan. 
\end{itemize}
\vspace{1cm}
\textbf{MF 1479\_2 Propagación de plantas en vivero}
\begin{questions}
  % 1
\question Señala cuale de las siguientes especies son coníferas
\begin{checkboxes}
  \choice A. Plataneros, encinas y robles
  \choice B. Pinos, cedros y abetos
  \choice C. Enebros, sabinas y cipreses
  \CorrectChoice D. Las respuestas B y C son correctas
\end{checkboxes}
% 2
\question ¿Qué ecosistemas son extremadamente sensibles a la contaminación por
  plantas invasoras y con las que hay que hay que extremar precauciones?
  \begin{checkboxes}
    \choice A. Ecosistemas de montaña
    \CorrectChoice B. Ecosistemas ede agua dulce (rios y sus
    riveras, lagos, humedales, etc)
    \choice C. Ecosistemas dunares
    \choice D. Ecosistema forestal
  \end{checkboxes}
% 3
\question ¿Como se llama la parte de la flor donde se forma y almacena el polen?
  \begin{checkboxes}
    \choice A. Caliz
    \CorrectChoice B. Estambre
    \choice C. Tubo polínico
    \choice D. Gineceo
  \end{checkboxes}
% 4
\question Los nutrientes de un suelo se clasifican en macroelementos y microelementos. ¿A
  qué se debe el nombre de estos últimos?
  \begin{checkboxes}
    \choice A. A qué la mayoría de elementos son de pequeño tamaño
    \choice B. A qué tienen poca importancia para las plantas
    \choice C. A qué se encuentran en el suelo en poca cantidad
    \CorrectChoice D. A qué se encuentran en las plantas en poca cantidad 
  \end{checkboxes}
% 5
\question Las fertilizaciones en un suelo pueden ser minerales u orgánicas. ¿Qué tipo
  de fertilizantes se emplean en la fertilización orgánica?
  \begin{checkboxes}
    \choice A. Estiércol, humus de lombriz o NPK inorgánico
    \choice B. Abono verde  o enmiendas calizas
    \CorrectChoice C. Estiércol, humus, compost, guano, gallinaza, abono verde
    \choice D. Las respuestas A y B son correctas
  \end{checkboxes}
% 6
\question El objetivo principal de la preparación de suelos es provocar transformaciones
  que mejoren la germinación y el desarrollo de las plantas. ¿Las preparaciones que se
  realizan pueden conseguir fines como?
  \begin{checkboxes}
    \choice A. Aireación del suelo y/o destrucción de hierbas no deseadas
    \choice B. Aportaciones de nutrientes o enmiendas para mejorar la calidad del suelo
    \choice C. Eliminación de actividad microbiana
    \CorrectChoice D. Las respuestas A y B son correctas
  \end{checkboxes}
% 7
\question ¿Qué propiedad física del suelo depende del tamaño de las partículas que la
  componen?
  \begin{checkboxes}
    \CorrectChoice A. Textura
    \choice B. Porosidad
    \choice C. Estructura
    \choice D. Ninguna respuesta es correcta
  \end{checkboxes}
% 8
\question ¿Las propiedades del suelo las podemos dividir en?
  \begin{checkboxes}
    \choice A. Físicas, químicas y texturales
    \CorrectChoice B. Físicas, químicas y biológicas
    \choice C. Ph, conductividad eléctrica y capacidad de intercambio catiónico
    \choice D. Las respuestas A y C son correctas
  \end{checkboxes}
% 9
\question El nombre científico de una especie se forma de dos partes. ¿Puedes
indicar a que categoría taxonómica corresponde la primera parte?
\begin{checkboxes}
  \choice A. Al Reino
  \choice B. A la Familia
  \CorrectChoice C. Al Género
  \choice D. Ninguna es correcta
\end{checkboxes}
\question La frase: \emph{``La ropa de trabajo corriente es un EPI fundamental''}, ¿es?
  \begin{checkboxes}
    \CorrectChoice A. Falsa
    \choice B. Falsa. Solo es un EPI fundamental si los pantalones son largos
    \choice C. Verdadera
    \choice D. Verdadera solo si el operario la utiliza correctamente
  \end{checkboxes}

\question ¿Los tipos de riesgos que un operario corre por  realizar tareas de abonado del
  terreno pueden ser?
  \begin{checkboxes}
    \choice A. Sobreesfuerzos por manipular cargas o posturas inadecuadas
    \choice B. Contacto con agentes químicos o ingestión accidental de tóxicos
    \choice C. Lesiones en la piel por salpicaduras de residuos o agentes químicos
    \CorrectChoice D. Todas las respuestas son correctas
  \end{checkboxes}

\end{questions}}

\end{document}
%%% Local Variables:
%%% mode: latex
%%% TeX-master: t
%%% End:
