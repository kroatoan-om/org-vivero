% Created 2019-05-13 lu. 15:02
% Intended LaTeX compiler: pdflatex
\documentclass[a4paper,12pt,oneside]{article}
\usepackage[main=spanish, english, ]{babel}%paquete para el idioma del documento. Si
%se quiere utilizar un parrafo con idioma diferente podemos utilizar
%la orden  electlanguage{}
\usepackage[utf8]{inputenx}
\usepackage[T1]{fontenc}
\usepackage{lmodern,pifont}
\usepackage{pdflscape}
\usepackage{caption}
\usepackage{textcomp}
\usepackage{graphicx}
\usepackage[dvipsnames]{color}
\usepackage{colortbl}
\usepackage{longtable}
\usepackage{hyperref}
\hypersetup{bookmarksopen,bookmarksnumbered,bookmarksopenlevel=4,%
  linktocpage,colorlinks,urlcolor=black,citecolor=ForestGreen,linkcolor=black,filecolor=black}
\usepackage{natbib}
\usepackage{amssymb}
\usepackage{amsmath}
\usepackage{geometry}
\geometry{a4paper,left=2cm,top=2cm,right=2.5cm,bottom=2cm,marginparsep=7pt, marginparwidth=.6in}
\usepackage[utf8]{inputenc}
\usepackage[T1]{fontenc}
\usepackage{graphicx}
\usepackage{grffile}
\usepackage{longtable}
\usepackage{wrapfig}
\usepackage{rotating}
\usepackage[normalem]{ulem}
\usepackage{amsmath}
\usepackage{textcomp}
\usepackage{amssymb}
\usepackage{capt-of}
\usepackage{hyperref}
\author{Antonio Soler Gelde. IT Forestal}
\date{}
\title{Propagación de plantas en vivero}
\hypersetup{
 pdfauthor={Antonio Soler Gelde. IT Forestal},
 pdftitle={Propagación de plantas en vivero},
 pdfkeywords={},
 pdfsubject={},
 pdfcreator={Emacs 25.3.1 (Org mode 8.2.10)}, 
 pdflang={Spanish}}
\begin{document}

\maketitle
\tableofcontents


\section{Aspectos básicos}
\label{sec:orgaf7fcae}
\subsection{Sistematica}
\label{sec:org2f98878}
\subsection{Claves dicotómicas y otros procesos de identificación}
\label{sec:orgff178b2}
\subsection{Plantas más frecuentes en vivero}
\label{sec:org35e0851}
\subsection{Organografía y fisiología vegetal}
\label{sec:orge5d2e27}
\subsubsection{Principales organos de las plantas (organografía)}
\label{sec:orga7c6b58}
\subsubsection{Operaciones de cultivo}
\label{sec:orgdd7ccda}
\subsubsection{Principales características fisiologicas}
\label{sec:org0105c8e}
\section{Preparación del medio de cultivo}
\label{sec:org241d138}
\end{document}