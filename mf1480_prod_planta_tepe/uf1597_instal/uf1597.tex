% Created 2019-05-26 dom 12:21
% Intended LaTeX compiler: pdflatex
\documentclass[a4paper,12pt,oneside]{article}
\usepackage[main=spanish, english, ]{babel}%paquete para el idioma del documento. Si
%se quiere utilizar un parrafo con idioma diferente podemos utilizar
%la orden  electlanguage{}
\usepackage[utf8]{inputenx}
\usepackage[T1]{fontenc}
\usepackage{lmodern,pifont}
\usepackage{pdflscape}
\usepackage{caption}
\usepackage{textcomp}
\usepackage{graphicx}
\usepackage[dvipsnames]{color}
\usepackage{colortbl}
\usepackage{longtable}
\usepackage{hyperref}
\hypersetup{bookmarksopen,bookmarksnumbered,bookmarksopenlevel=4,%
  linktocpage,colorlinks,urlcolor=black,citecolor=ForestGreen,linkcolor=black,filecolor=black}
\usepackage{natbib}
\usepackage{amssymb}
\usepackage{amsmath}
\usepackage{geometry}
\geometry{a4paper,left=2cm,top=2cm,right=2.5cm,bottom=2cm,marginparsep=7pt, marginparwidth=.6in}
\usepackage[utf8]{inputenc}
\usepackage[T1]{fontenc}
\usepackage{graphicx}
\usepackage{grffile}
\usepackage{longtable}
\usepackage{wrapfig}
\usepackage{rotating}
\usepackage[normalem]{ulem}
\usepackage{amsmath}
\usepackage{textcomp}
\usepackage{amssymb}
\usepackage{capt-of}
\usepackage{hyperref}
\author{Antonio Soler Gelde. IT Forestal}
\date{}
\title{Propagación de plantas en vivero}
\hypersetup{
 pdfauthor={Antonio Soler Gelde. IT Forestal},
 pdftitle={Propagación de plantas en vivero},
 pdfkeywords={},
 pdfsubject={},
 pdfcreator={Emacs 25.3.1 (Org mode 8.2.10)}, 
 pdflang={Spanish}}
\begin{document}

\maketitle
\tableofcontents


\section{Uso de instalaciones, maquinaria y equipos}
\label{sec:orgcd55230}
\subsection{El vivero}
\label{sec:org1f22cdc}
Definimos \textbf{vivero} como \emph{un lugar o terreno dedicado a la multiplicación u cría
de plantas hasta que puedan ser vendidas o plantadas en su lugar definitivo}. 
\subsubsection{Factores para su ubicación}
\label{sec:org9b26fd6}
A la hora de ubicar un vivero hay que considerar diferentes factores, los cuáles
resumimos a continuación:
\begin{itemize}
\item \textbf{Topografía:} Preferibles lugares con \uline{poca pendiente} que hagan fácil la
mecanización e incluso evitar los  fondos de valle para minimizar las heladas
tempranas y tardías. Generalmente \uline{orientación} norte-sur para recibir
radiación solar de manera uniforme durante todo el año.
\item \textbf{Vías de acceso:} A ser posible la ubicación ha de estar cerca de vías de
acceso importantes para que puedan acceder camiones grandes. Esto \uline{abarata}
los costes de los portes.
\item \textbf{Agua:} El cultivo en un vivero \uline{exige riego}, lo que hace necesario poder
disponer de este recurso en \uline{cantidad y calidad} suficiente.
\item \textbf{Clima:} El clima \uline{influye en el crecimiento}. Un clima con un largo período de
crecimiento, ausencia de heladas fuera de temporada y de sin fuerte calor en
verano sería ideal.
\item \textbf{Suelo:} Este factor tendrá mucha importancia en el caso de los \uline{viveros a
raíz desnuda}. En los casos de cultivo en \uline{envase o contenedor} La composición
del suelo no es importante ya que el sustrato se adaptará a los
requerimientos del vivero.
\item \textbf{Mano de obra:} Depende del las operaciones que se vayan a mecanizar, pero en
un vivero siempre hay trabajos que pueden ser mas o menos puntuales y
requieren  de bastante mano de obra: semillado, esquejado, repicado, etc
\end{itemize}
\subsubsection{Tipos de vivero}
\label{sec:org6fcfe73}
Podemos clasificar los viveros según varios criterios:\\
1- \textbf{Según el lugar donde se cultivan las plantas:}
\begin{itemize}
\item Viveros al exterior o a raíz desnuda: Plantas estarán cultivadas en el propio
suelo del vivero, al aire libre o dentro de un invernadero.
\item Viveros en contenedor: Las plantas estarán en envases, en invernaderos o
túneles. Antes de pasar al exterior necesitaran un \uline{periodo de
endurecimiento}.
\end{itemize}
2- \textbf{Según el destino de las plantas:}
\begin{itemize}
\item Vivero comercial: Produce plantas para ser vendidas. Puede ser venta al por
mayor, al detalle o ambas.
\item Vivero privado: Las plantas producidas son utilizadas para fines privados, por
ejemplo una repoblación propia, investigación, etc.
\end{itemize}
3- \textbf{Según permanencia:}
\begin{itemize}
\item Vivero provisional: Se construyen para abastecer de planta a la repoblación o
plantación de una determinada zona.
\item Vivero fijo o permanente: Construidos para que perduren en el tiempo. La
mayoría de viveros que abastecen el mercado son \uline{permanentes}.
\end{itemize}
4- \textbf{Según especialización:}
\begin{itemize}
\item Vivero general: Plantas de distintas especies y tamaños
\item Vivero especializado: Tiene una gama más reducida de especies. Pueden ser
forestales, frutícolas, hortícolas, ornamentales, etc
\end{itemize}
\subsubsection{Distribución de espacios}
\label{sec:org461cd66}
En un vivero podemos distinguir tres espacios principales
1- \textbf{Sección de germinación:}
Zona del vivero dedicada a la realización de semilleros para su
germinación. Esta zona dependiendo de la técnica de cultivo y del clima puede
ser al aire libre, bajo malla de sombreo o en invernadero. 
Podemos encontrar diferente equipamiento en estas zonas como por ejemplo:\\
\begin{itemize}
\item Camas de cultivo
\item Bancales
\item Zonas de barbecho
\item Cámaras de germinación
\item Túneles de propagación
\item Mesas de cultivo\\
\end{itemize}

2- \textbf{Sección de crecimiento o plantel:}
Formada por las zonas del vivero en las que se deposita la planta para su
desarrollo.
3- \textbf{Elementos complementarios o auxiliares:}
Son todos los espacios no cultivados o auxiliares. Entre otros:\\
\begin{itemize}
\item Red viaria
\item Cerramientos
\item Naves para acopio de materiales o trabajo
\item Oficinas
\end{itemize}
\subsubsection{}
\label{sec:org5237ad5}
\end{document}